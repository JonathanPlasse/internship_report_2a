\chapter{State of the Art: Frameworks to Control Aerial Robot Swarms}

In this chapter, different autopilot frameworks, flight controllers, and onboard computers are studied to choose the best fitting for this project.

\section{Presentation of Autopilot Frameworks}
    \subsection{Ardupilot}
    ArduPilot is an open-source autopilot framework since 2009.
    It can run on different types of vehicles, not only drones but also rover, helicopter, and even submarine.
    It is full-featured and reliable. It has thousands of hours of flight over 1 million vehicles worldwide.

    \subsection{PX4}
    PX4 is based on ArduPilot and now has a community of its own. It is more aim toward a commercial solution.
    It is backed by the Linux Foundation.
    Generally, PX4 and Ardupilot have similar features.

    \subsection{ROSflight}
    The core strength of ROSflight is that it has been designed to control a drone with a companion computer
    and it has an easy-to-understand code base that is kept lean.
    It is easy to build on top of ROSflight to add functionalities. The project ROScopter \cite{github_roscopter}, for example, adds the support of waypoint position control.

\section{Criteria}
The chosen autopilot framework will be the one that respects most of the criteria.

    \subsection{Implementation}
    Different features are needed for this project; the one that needs the less further implementation is preferred.

        \subsubsection{Direct control of motor speed}
        The control algorithms are to be able to control the motor speed of the drone directly.

        This functionality could be implemented on all autopilots \cite{github_ardupilot_11859}\cite{px4_low_level_control}\cite{github_rosflight_98}.

        \subsubsection{EKF}
        The Extended Kalman Filter (EKF) is an algorithm to merge different sensor data and obtain the drone's position and orientation, among other measures.
        
        On both Ardupilot and PX4, the EKF runs on the flight controller, so nothing has to be done.
        ROSflight runs the EKF on the onboard computer, not on the flight controller, so it needs to share data between the two, which is inefficient.
        Also, the EKF on ROSflight takes less information into account.

        \subsubsection{Compatible with Vicon}
        The autopilot should be able to use the motion capture data in the EKF.

        For Ardupilot and PX4, a MAVlink message can be used to transfer the data to the EKF.
        But for ROSflight, it has to be implemented.

    \subsection{Simulation}
    The Software In The Loop (SITL) is the idea of running the autopilot software inside a physical simulation.
    Thanks to this, the modification in the autopilot can be tested in software before the real test.
    It can avoid breaking the drones and allow them to test less in real life.

    \subsection{ROS interface}
    All autopilot frameworks are compatible with ROS. But ROSflight is the only one completely ROS based.

    \subsection{Open-source}
    All autopilot frameworks are open-source. The only difference is the licenses.

    \subsection{Community}
    Both Ardupilot and PX4 have huge communities that are ready to help.
    That is not the case with ROSflight, even if the developer of ROSflight is responsive.

\section{Hardware}

    \subsection{Flight controller}

        \subsubsection{PixHawk or PixRacer}
        The PixHawk is compatible with all autopilot.
        It is the flight controller use by everyone. There is much documentation. And ArduPilot recommend using it \cite{ardupilot_choose_fc}.
        A small version of it is the PixRacer that is more affordable and adapted for this project \cite{mrobotics_pixracer}.

        \subsubsection{OpenPilot Revolution}
        It is compatible with ROSflight and ArduPilot. It is a popular flight controller.

        \subsubsection{Navio2 with Raspberry Pi}
        It is compatible with Ardupilot and PX4. It is nice as it is used both as a flight controller and onboard computer.

    \subsection{Onboard computer}

        \subsubsection{Nvidia Jetson Nano}
        It is compatible with every autopilot, but it takes more place than the Raspberry Pi.

        \subsubsection{Raspberry Pi}
        It is compatible with every autopilot, and it takes not much space as it is used both as a flight controller and onboard computer.
        It is also widely used compared to the Nvidia Jetson Nano.
        It is less powerful than the Nvidia Jetson Nano. But the Raspberry 4 will soon arrive with good performances.

\section{Conclusion}
ROSflight demands a lot of features implementation compared to PX4 and Ardupilot.
PX4 and Ardupilot are similar, so either of them could be used.

So, the navio2 with the raspberry pi seems to be the solution that takes less space and is compatible with both PX4 and Ardupilot.
That is why it was chosen.
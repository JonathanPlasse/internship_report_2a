\chapter{State of the Art: Frameworks to Control Aerial Robot Swarms}
    
    For the following discussion it is assumed that the drone has an \gls{oc}.
    The aim is to test a computationally demanding low-level control algorithm directly controlling the speed of the motor of the drone. This is today impossible on most of the available flight controllers.\\
    Check out these blog post for a full comparison of open source drone software frameworks \cite{drone_software_projects} and many flight controllers \cite{drone_flight_controllers}.
    \section{ROSflight \& ROScopter}
    \subsection{Strength}
    The core strength of ROSflight is that it has been designed to control a drone with a companion computer, and it has an easy-to-understand code base that is kept lean.
    It is easy to build on top of ROSflight to add functionalities. The project ROScopter \cite{github_roscopter} for example add the support of waypoint position control.
    
    To use ROSflight for a swarm of drone, there would be a ground control station that run \texttt{roscore} and send the vicon data to every drone via wifi.
    Each drone has an \gls{oc} connected via wifi to the \acrlong{gcs} and have no radio receiver as it would be to expensive to have a pair receiver/transmitter for each drone. Or, bind the same transmitter to all the receiver, but then we cannot control any drone because they are linked to the same transmitter.
    The \gls{oc} would run a \texttt{rosflight\_io} node.
    The firmware safety about the RC will have to be disabled and other should be implemented to deactivate the drone if needed (a control on the \gls{gcs} keyboard or xbox 360 controller).
    
    ROScopter could be used to control the position of the drone with waypoints and use \gls{ekf} as sensor fusion algorithm instead of the simpler one from ROSflight. But would need modification to use vicon data instead of gps data.
    
    Then, there would be the direct control of each motor speed instead of controlling roll, pitch, yaw and thrust.
    
    There is a pull request \cite{github_rosflight_98} that can directly control from the onboard computer the PWM outputs of the flight controller.
    
    The \gls{oc} could communicate to the \gls{gcs} via wifi the data log (position, speed, attitude, motor speed, …) of the fly that would then be stored in a file.
    
    ROSflight also have a SIL simulator. It simulate the firmware of rosflight. And thanks to it, it is possible to test the drone control virtually before the real life test.
    
    There is not a list of flight controllers supported except for the OpenPilot Revolution some project have used PixHawk as \gls{fc}.
    So, if another flight controller is chosen, some more configurations would have to be done.
    
    ROSflight use usb connection between the flight controller and the \gls{oc}. A latency of the order of 10ms can be present {\color{red} please provide a reference for this 10ms. Also give this value for the other 2 frameworks so we can compare in fair way}. But, there is an untested solution \cite{github_rosflight_98}, that could reduce it.
    
    It is also possible to use the UART3 of the flight controller instead of usb but this is an undocumented and untested \cite{github_rosflight_98}.
    But the latency would be reduced. The bandwidth would be reduced compared to usb.
    {\color{red} I did not understand what you meant with: Most x86 computers don't have a UART peripheral accessible in the kernel --> i saw it in on eof the links you sent me.}
    Both Raspberry and jetson nano can use \texttt{/dev/tty<something>} as serial bus.
    
    So, to recapitulate. Usb has high bandwidth (3,000,000 baud) and would be enough for the project \cite{github_rosflight_98} but there is possibly a latency problem.
    And, serial has limited bandwidth compared to usb (115200 baud is standard but it could be increased. To Test) but no latency problem.
    
    
    Between the \gls{oc} and the \gls{fc} these informations could be exchanged:
    \begin{itemize}
        \item Raw data sensor
        \item \gls{ekf} estimation
        \item PWM command
        \item …
    \end{itemize}
    
    Between the \gls{oc} and the \gls{gcs} these informations could be exchanged:
    \begin{itemize}
        \item vicon data
        \item \gls{ekf} estimation
        \item PWM command
        \item for the distributed Model Predictive Control we will have to send information between agents
        \item …
    \end{itemize}
    
    The bandwidth needed for the PWM command is it at least
    2 bytes * 4 motors * 500 Hz * 8 bit = 32 kb/s to control PWM outputs.{\color{red}is what i the relation between baud and kb/s?}

    \section{ArduPilot}
        ArduPilot and more precisely ArduCopter has a lot of features.
        It has numerous flight modes.
        It has its own Ground Control Station to flash firmware and modify the settings.
        ArduPilot is compatible with a lot of flight controller
        It has a strong community.
        It has a really good documentation.
        It is tested a lot and the users report problem when they arise.
        
        There is no implementation at the moment to directly sent PWM command from the onboard computer to the motor.
        {\color{blue} Update: Apparently, it is possible by using \texttt{SERVO<X>\_FUNCTION=RCIN<Y>PASSTHROUGH} and \texttt{RC\_CHANNELS\_OVERRIDE}}
        \cite{github_ardupilot_11859}.
        
        For swarm, it would be the same as ROSflight but using \texttt{MAVROS} instead of \texttt{rosflight\_io}.
        
        ArduPilot also have a SITL simulator. It simulate the firmware of rosflight. And thanks to it, it is possible to test the drone control virtually before the real life test.
    
    \section{Dronecode/PX4}
        PX4 has a lot of features.
        It has numerous flight modes.
        \Gls{ekf} is computed on the flight controller.
        It has its own Ground Control Station to flash firmware and modify the settings.
        It has a strong community.
        It has a good documentation.
        It is tested a lot and the users report problem when they arise.
        
        There is a pull request to directly sent PWM command from the onboard computer to the motor but it has not been active for 5 months \cite{github_px4_10863}. It is not going to happen soon.
        There is also a possible solution that I posted here \cite{px4_low_level_control}.
        
        
        
        For swarm, it would be the same as ROSflight but using \texttt{MAVROS} instead of \texttt{rosflight\_io}.
        
        ArduPilot also have a SITL simulator. It simulate the firmware of rosflight. And thanks to it, it is possible to test the drone control virtually before the real life test.
    
    
    \section{Other Frameworks}
        \subsection{Frameworks related to motion capture}
            {\color{red}Please explain which compatibilities exist for the other frameworks and what they use to allow that \\}
            {\color{red}In crazyswarm they use an improved version of vicon bridge \cite{ros_wiki_vicon_bridge}, so it would be interesting if you could use the same (also compatible with other mocap systems like optitrack). Here \cite{github_usc_actlab} you can find the libmotioncapture and the libobjectracker. In crazyswarm project you can see how they implemented all required functionality. I would focus first on making e.g. ROSflight work with libmotioncapture (using the tracker of Vicon, which works the best). The other one, i.e. libobjectracker, is not needed for this project because it is only useful for very small uavs as the crazyflie (because of the limited space to put markers). So you have an idea where you can find the code.\\}
            
    \section{A comparison between the different Frameworks}
        Every frameworks has a simulator to test the control algorithm.
        mavlink to communicate between the companion board and the flight controller, failsafe
        
        PX4 and ArduPilot are extremely similar.
        
        \subsection{Communication}
            The communication between the \gls{oc} and the \gls{fc} of the drones are done with the MAVlink protocol. All firmware use MAVlink but each firmware has its own implementation. ArduPilot and PX4 use MAVROS and ROSflight use rosflight\_io 
            
        \subsection{Network}
            There is one ground control station and the \gls{oc} of the drones are connected to it via wifi. 
        
        \subsection{ROS}
            About ROS, there is only one roscore needed on a network \cite{stackoverflow_multi_machine}.
            So, the ground control station run roscore and stream vicon data to the drones. It would also create log file.
            The \gls{oc} on the drones would run either \texttt{MAVROS} or \texttt{rosflight\_io}.
            It would also run the control algorithm.
            {\color{red} \\Please also discuss the current and coming capability for ROS2 which would be ideal for real-time and multi-robot system. I think PX4 is most advanced in this, see project status roadmap ref project status \cite{px4}, also ref \cite{ros_wiki_ng_drones}. I think the advantages that ROS2 brings are of a great asset for computation and communication for the need of this project.\\}
            
            There is also ROS2, 
        
        \subsection{Vicon}
            I did not found any project that used Vicon.
            \texttt{vrpn\_client\_ros} or \texttt{libmotioncapture}.
        
            Maybe \texttt{vrpn\_client\_ros} would be more adapted to our project as it use ros or ros package could be created around \texttt{libmotioncapture}.
            {\color{red}a Vicon bridge to ROS is what you need. Look to how they did it in crazyswarm.}
        
        \subsection{Extended Kalman Filter}
            ArduPilot and PX4 have the \gls{ekf} on the flight controller.
            The MAVLINK message \texttt{ATT\_POS\_MOCAP} has been to created to give the data of Vicon to the \gls{ekf}.
            
            ROSflight has just a complementary filter on the flight controller (a simpler algorithm).
            But, it is possible to run an \gls{ekf} on the \gls{oc} with ROScopter but it would have to be modify to add vicon support.
            
            Running the \gls{ekf} on the \gls{fc} means that vicon data has to be sent to the \gls{fc} and then the \gls{ekf} estimation has to be sent back to the \gls{oc}.
            
            Running the \gls{ekf} run on the \gls{oc} means that the \gls{fc} has to sent the sensor data to the \gls{oc} and optionally sent back the \gls{ekf} estimation to the \gls{fc} if we want to use the \gls{fc} as controller.
            
            For the second solution, the EKF needs to be implemented and tuned.
            The first solution is easier to set up and would be ideal if we want to compare the performance of our own control algorithm to the PX4 and ArduPilot numerous flight mode but ROSflight would not have a position estimate from Vicon.
            
        \subsection{Sending PWM command}
            ROSflight is the most advanced, in the implementation.
            {\color{red}Explain how the solution of raspberry pie and the emlid navio2 deck. Would there be less latency and bandwidth problems if evrything is running on the rasppie 3/4? Just having all computation in 1 unit seems very interesting to me.}
        
        \subsection{Swarm compatibility}
            Both ArduPilot \cite{ardupilot_multi_vehicle} and PX4 \cite{px4_multi_vehicle}
            have proof of concept that swarm of drone are possible.
            
            It should be possible for ROSflight \cite{github_rosflight_100} too.
            But, some firmware modification has to be done.
        
        \subsection{Simulator}
            All firmware have SIL/SITL simulator.
            Only PX4 can simulate multiple drone out of the box {\color{red}and even in ROS2!}. I asked the question \cite{px4_ros2}.
            
        \subsection{Serial bus}
            Both Jetson Nano and Raspberry Pi have a hadware serial port.\\
            ROSflight \cite{rosflight_parameters},
            ArduPilot \cite{ardupilot_serial_parameters},
            PX4 \cite{px4_serial_parameters},
            can all configure their baud rate at least up to 115200. That is a baudrate that 
            Jetson Nano \cite{jetsonhacks_serial}
            and
            Raspberry Pi support \cite{rpi_stackexchange_serial}.
            Higher baud rate can be tested when the flight controller has arrived.
        
        
        
        \subsection{Flight controller}
            \subsubsection{OpenPilot Revolution}
                It is compatible with ROSflight and ArduPilot. It is a popular flight controller.
                It comes with all the connectors needed \cite{bangood_revo}.
        
        
            \subsubsection{PixHawk 2.4.8/PixRacer}
                The PixHawk is compatible with every firmware.
                It is the \gls{fc} use by everyone, there is a lot of documentation, project. And ArduPilot recommend using it \cite{ardupilot_choose_fc}.
                A small version of it is the PixRacer that is more abordable and adapted for this project \cite{mrobotics_pixracer}.
        
            \subsubsection{Navio2+Raspberry Pi 4}
                As the Raspberry Pi 4 has just been released it has different problems that has not been solved such as overheating, firmware problems… It is not software compatible with the NAVIO2 at the moment, they will release an new image when they will have done some test. It is adviced to wait until it had a bit of time on the market. So for now the raspberry pi 4 is not reliable enough for this project \cite{ardupilot_rpi_compatibility}.
        
            If we use the Raspberry Pi 3B+ with Navio2, the frequency update is limited to 400Hz \cite{emlid_pwm_frequency}.
            
            Here is a blog post about Navio2 + Raspberry Pie \cite{dojofordrones_rpi_drone}.
            
            It is not recommended to directly control pwm and at the same time running ArduPilot \cite{emlid_servo_control}. It is recommended to control pwm via ArduPilot. It has to communicate via MAVLINK like the other \glspl{fc}.
        
        \subsection{On-board Computer}
            \subsubsection{Nvidia Jetson Nano}
        
            \subsubsection{Raspberry Pi}
    
    \section{Conclusion}
        ROSflight is a lightweight solution. The bare minimum is implemented, everything else has to be implemented and ROScopter add some functionality. But, it cannot compare to the full-fledged ArduPilot and PX4 \cite{reddit_firmware_ros}. Moreover, when somebody asked on forum if they should use ROSflight, it was recommended to use either ArduPilot or PX4 \cite{reddit_swarm_ros}.
        
        In term of hardware, RaspberryPi4 + Navio2 is the easiest solution to directly control PWM outputs 
        
        {\color{red}For me this conclusion is partially correct. I feel like the raspberry pie + navio idea is missing. }
        {\color{red}I feel more that ArduPilot and PX4 are so advanced that it would take somuch time to get even just some of the functionality in ROSflight. also the fact that e.g. with ArduPilot there are many types of systems we can use it very interesting for multi-robot. So I would put focus on ArduPilot/PX4 since they are very similar. But then the solution of a small rasppie 3 (and later 4) with navio 2 deck would be very relevant. also to add other motors for uav with arm for example we can use pins of rasppie. And rasppie is used by so many people.}
\chapter{Linux}

\section{Environment Variables}
An environment variable is a dynamic-named value that can affect the way running processes will behave on a computer \cite{wikipedia_env_var}.
\subsection{ROS}
In this project we modified different ROS environment variables \cite{ros_env_var}.
\subsubsection{ROS\_MASTER\_URI}
It specifies where the ROS master can be find.
It is only needed for multiple devices.

\subsubsection{ROS\_IP ou ROS\_HOSTNAME}
It tells to the ROS master what is the IP address or the hostname of the computer.
It is only needed for multiple devices.

\section{Services} \label{services}
A detailed explanation is available here \cite{techrepublic_services}.
To control the different services you can use these commands.
\begin{description}
    \item[sudo systemctl start/stop/restart <service>] to start/stop/restart a service.
    \item[sudo systemctl enable/disable <service>] to enable/disable a service at start-up.
\end{description}

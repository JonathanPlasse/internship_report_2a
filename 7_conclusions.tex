\chapter{Conclusions and Future Work}

First, the different autopilot frameworks were researched to choose the most appropriate one.
Once the autopilot framework was chosen, it was the turn of the hardware of the drone.
Then, a first drone was assembled and tested.
With a functioning drone, it was then time to start implementing the different functionalities for this project.


During this internship, different autopilot frameworks were researched to choose the most appropriate one.
Ardupilot was chosen but it could be possible to go to PX4 as everything is compatible.

A general guide has been done to design a drone. It has been used to create the drones of this project.
And, it could be reused to redesign another drone if needed.

Every step of assembling and set up has been detailed to be reproductible.

The first test of the drone was successful.
So, the different features needed for the final project started to be implemented.

The set up of ROS on multiple machine worked.
The simulation has been done for one drone but could normally be done with all drones.

The interface with Vicon was not done but it should not be difficult to set it up.

The most important step to be done is the ROS nodes for the communication between the drones and running the control algorithms.
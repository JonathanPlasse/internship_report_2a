\chapter{Introduction}\pagenumbering{arabic}

\section{Presentation of BruBotics}

BruBotics is a team of researcher from the mechanical engineering departement of the Vrije Universiteit Brussel (VUB).

BruBotics is mainly oriented toward mechatronics, robotics, prosthesis, exoskeleton, and recently self-healing materials.

My internship mentor, Bryan Convens, has gone in a new direction for his Ph.D. thesis.
He has the project to fly a swarm of drones that can dynamically avoid each other.

\section{Aim of this internship}

The goal is to create a platform (the hardware and software) that can be used to fly a swarm of drones.
With this platform, Bryan Convens will be able to test different control algorithms for his thesis.

For this project, the drone has to have high bandwidth communication to communicate information between them
and avoid each other.
Motion Capture is used as it gives precise position and orientation of the drones. It is a solution that is already used at BruBotics.
Each drone has to have an onboard computer as the control algorithm is computationally demanding.

This an ambitious project as there has not been any open source project with all the requirements presented.

Other project requirements are that the project should be as open-source as possible to facilitate replicability.
If possible, it should use the ROS framework to facilitates sharing robotics software and is widely used in the world of research.

The use of existing solutions should be prioritized as it minimizes the complexity and development time of the project

The drone should be as small as possible to allow a maximum of drone (5 to 30 drones) to fly as it has to fly indoors and, the motion capture area is limited.

It should also be possible to simulate the swarm in software to test the control algorithm before the real flight.

\section{Approach}

First, the different autopilot frameworks were researched to choose the most appropriate one.
Once the autopilot framework was chosen, it was the turn of the hardware of the drone.
Then, a first drone was assembled and tested.
With a functioning drone, it was then time to start implementing the different functionalities for this project.

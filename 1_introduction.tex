\chapter{Introduction}\pagenumbering{arabic}

{\color{red}
\begin{itemize}
    \item Present VUB
    \item Present the motivation of the project
    \item Present what is to be done in the project
    \item Present the Requirements
    \item \dots
\end{itemize}
}

\section{VUB}
The {\color{red}Vrje Universitat Brussel} (VUB) is

BruBotics a team inside of VUB, is mainly oriented toward mecatronics. The core of competence of the group is prostesis.
Bryan Convens has gone in a new direction for his PhD.
He has for project to fly a swarm of drone that avoid each other online.


A motion capture system (Vicon) will be used and high bandwidth communication will be necessary for the drone.
Two possibilities is possible, use an out-of-the-box drone or designed a custom one for this project.

A study of different software solution will be studied,
Autopilot firmware,
Software to run on the offboard computer. (ROS)
Software that run on ground station.




The idea of this internship has been to create a platform to test control algorithm on a swarm of drones using high bandwidth wireless communication, motion capture, and ROS.

\section{Project Requirements}
\begin{itemize}
    \item The project should be as open source as possible.
    \item If possible, it should be ROS based.
    \item It should use as much as possible existing solution, to minimize development time.
    \item It should work with quad-copter. Drone configuration is optional.
    \item It should fly indoors
    \item It should use Vicon Motion Capture system
    \item It should have a dynamic simulation of the swarm
    \item It should have high bandwidth wireless communication.
    \item It should be able to fly a swarm of 5 to 30 drones.
    \item The drone should be as small as possible.
\end{itemize}

\section{Approach}
Exploring the possible solutions of autopilot software to see which one would be the best fit for this project.

Designing a drone that can fly with a companion computer and assuring it is compatible with the chosen autopilot software.

Building a drone and creating instructions on how to build the drones.

Configuring and testing a drone without making a change to the autopilot software.

Making extension to the project to use multiple drones, simulation and motion capture.
